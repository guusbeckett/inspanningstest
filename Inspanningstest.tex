\documentclass[]{article}

\usepackage{amssymb,latexsym,amsmath}     % Standard packages

\addtolength{\textwidth}{1.0in}
\addtolength{\textheight}{1.00in}
\addtolength{\evensidemargin}{-0.75in}
\addtolength{\oddsidemargin}{-0.75in}
\addtolength{\topmargin}{-.50in}

\begin{document}

\title{Inspanningstest}

\author{\textbf{23TI2A1}\\Jim van Abkoude\\Guus Beckett }
\maketitle
\newpage
\tableofcontents
\newpage
\section{Inleiding}
Met een inspanningstest wordt uw uithoudingsvermogen en getraindheid   nauwkeurig vastgelegd. Daarnaast geeft deze inspanningstest belangrijke informatie over hoe uw hart, spieren en longen (samen)werken.
\newpage
\section{\r{A}strandtest}
\subsection{Warming-up}
De \r{A}strandtest begint met een warming up, tijdens deze warming up begint men te fietsen met een weerstand in de vorm van een kracht van 50 watt op de rem. Bij vrouwen is dit een weerstand van 25 watt. Tijdens deze warming up moet de fietser proberen een constant aantal omwentelingen per minuut te halen (60). De weerstand van de rem wordt daarna opgevoerd totdat de fietser een hartslag heeft tussen 120 tot 170 slagen per minuut.
\subsection{De test}
Wanneer de warming-up uitgevoerd is begint de test van zes minuten. De fietser moet ook tijdens deze test proberen zijn aantal omwentelingen per minuut rond de 60 te houden. Na zes minuten wordt de hartslag gemeten en daarna mag de fietser op eigen tempo uitfietsen.
\subsection{Resultaat}
Om te berekenen wat de conditie is van de fietser moeten er nu een aantal berekeningen gedaan worden. Hiervoor zijn nodig:
\begin{itemize}
\item De gemeten hartslag van de fietser aan het eind van de test
\item Het gewicht van de fietser
\item De leeftijd van de fietser
\item Een tabel die de relatie tussen de hartslag en het gebruikte wattage
\end{itemize}
\newpage
\section{Toepassing}
\newpage
\end{document}